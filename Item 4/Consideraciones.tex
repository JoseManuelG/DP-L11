\documentclass[a4paper]{article}

\usepackage[spanish]{babel}
\usepackage[utf8]{inputenc}
\usepackage{xcolor}
\usepackage[hyphens]{url}
\usepackage[colorlinks=true, urlcolor=black]{hyperref}

\definecolor{shadecolor}{RGB}{220,220,220}

\title{Consideraciones}

\date{}

\begin{document}
\setlength{\voffset}{-1in}
\setlength{\textheight}{680px}
\setlength{\headsep}{30px}
\pagenumbering{gobble}
\maketitle

Aqui se presentan los casos de test implementados. La documentación pertinente está en el propio código. También se aportan en el item 3 para que puedan ser ejecutados.

Los test se han realizado siguiendo solo parcialmente el esquema dado en teoría en esta asignatura, como se explicó en el entregable anterior. Para evitar los problemas que surgían a raíz de usar el driver, se llama a la plantilla repetidas veces con distintos parámetros en distintos test unitarios dentro de la misma clase.

\end{document}