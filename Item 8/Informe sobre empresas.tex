\documentclass[a4paper]{article}

\usepackage[spanish]{babel}
\usepackage[utf8]{inputenc}
\usepackage{xcolor}
\usepackage[hyphens]{url}
\usepackage[colorlinks=true, urlcolor=black]{hyperref}
\usepackage{graphicx}

\definecolor{shadecolor}{RGB}{220,220,220}

\title{Informe sobre empresas}

\date{}

\begin{document}
\setlength{\voffset}{-1in}
\setlength{\textheight}{680px}
\setlength{\headsep}{30px}
\maketitle

\section{Introducción}

Encontrar una empresa en el entorno local que se ajuste a tu perfil es un trabajo que en muchas ocasiones resulta difícil. Con la ayuda de herramientas de búsqueda de trabajo como la red LinkedIn podemos encontrar puestos que contengan algunas tecnologías, sin embargo, en muchas ocasiones son consultoras las que ofertan estos puestos, sin dar a conocer de forma explícita para que empresa es la oferta. Para elegir que empresas pueden sernos de más interés hemos hablado con algunos contactos que ya se mueven en el mundo empresarial y las hemos comparado con las ofertas encontradas.

\section{Empresas}

Hemos seleccionado cuatro empresas que buscan nuevos empleados, una en el entorno local, dos en el nacional y una fuera de España.

\subsection{Indra}
Indra es una empresa global, que abarca una gran cantidad de sectores. Hemos encontrado ofertas de trabajo de Indra en la que requieren conocimientos de Spring en Valencia como desarrollador junior. Aunque en esta oferta no podemos encontrar para que proyectos se trabajaría, si que podemos encontrar algunos de sus proyectos en la zona de prensa de su página, entre otros la digitalización de la administración pública italiana y un sistema inteligente de gestión de energía para el Sena. Podemos intuir, por tanto, que esta empresa usa Spring como apoyo a sus sistemas de información, dando funcionalidad a clientes de sus proyectos, y si bien no podemos saber que tecnología exacta usan en estos proyectos de Spring, algunos de los A+ que hemos realizado podrían ser de gran utilidad dentro de este entorno (como por ejemplo test funcionales parametrizados o el acceso a la base de datos a través de repositorios paginados).

\subsection{Emergya}
Emergya es una empresa con origen nacional, pero que opera actualmente de manera global. Podemos ver en su web ejemplos de los proyectos en los que han trabajado, incluyendo un apartado de proyectos web, como el portal de Unicef. También hemos encontrado un puesto de trabajo de esta empresa en Sevilla, en el que uno de los requisitos es tener conocimientos de Spring. Como en el caso anterior, tenemos muchas posibilidades de poder alinear nuestros conocimientos de los A+ realizados con el modo de trabajar de esta empresa.

\subsection{Everis}
Esta también es una empresa global, perteneciente a uno de los mayores grupos de IT, que actualmente busca desarrolladores en Madrid. También es un requisito imprescindible conocer Spring, y por tanto creemos que sus proyectos pueden estar muy relacionados con nuestros conocimientos. Entre sus servicios encontramos algunos como \textit{New Digital Business} o \textit{Business Aplications}

\subsection{Sngular}
Sngular es otra multinacional, de un tamaño menor. Como las anteriores, esta también tiene actualmente oferta de trabajo para desarrolladores con conocimientos de Spring, en Texas, Estados Unidos. Su web ofrece una visión de algunos proyectos destacados de éxito, como una plataforma para BBVA y otra para REDIX Gas. Creemos que nuestros conocimientos también encajarían para poder empezar en esta empresa.

\section{Tecnología}
Hemos elegido cuatro empresas que usan Spring, y por tanto, muchos de los A+ que hemos desarrollado ya podrían estar alineados con estas empresas, como hemos comentado anteriormente. Algunos puede que nos resultaran más o menos útiles en una empresa de gran envergadura como estas, y de entre los posibles A+ que podríamos realizar queremos destacar un A+ de una tecnología que, si no usan las cuatro empresas, seguro que alguna de ellas si que lo usan. Para empresas de este tamaño es sin duda importante asegurar que los servidores donde desplieguen sus aplicaciones aguantan el tráfico al que se van a someter. Por ello, los test de rendimiento deben hacerse de manera distribuida y no desde la propia máquina de preproducción, realizando una simulación más cercana a la realidad. 


\end{document}